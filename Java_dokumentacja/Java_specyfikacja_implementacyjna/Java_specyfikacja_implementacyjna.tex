\documentclass[a4paper,12pt,oneside]{article}

\usepackage[utf8]{inputenc}
\usepackage[pdftex]{graphicx}
\usepackage{polski}
\usepackage{amsfonts}
\usepackage{verbatim}
\usepackage{indentfirst}
\usepackage{listings}
\usepackage[polish]{babel}
\usepackage[T1]{fontenc}
\usepackage{fancyhdr}
\usepackage{lastpage}
\pagestyle{fancy}
\renewcommand{\headrulewidth}{0pt}
\rhead{}
\lhead{}
\cfoot{str. \thepage/\pageref{LastPage}}
\newcommand{\multlinecomment}[1]{\directlua{-- #1}}

\begin{document}
\makeatletter
\addtocontents{toc}{\protect\thispagestyle{empty}}
\newcommand{\linia}{\rule{\linewidth}{0.4mm}}

\renewcommand{\maketitle}{\begin{titlepage}

    \vspace*{1cm}

    \begin{center}\small

    Politechnika Warszawska\\

    Wydział Elektryczny\\

    \end{center}

    \vspace*{1cm}

    \noindent\linia

    \begin{center}

      \LARGE \textsc{\@title}

         \end{center}

      \noindent\linia

    \vspace{1cm}

    \begin{flushright}

    \begin{minipage}{5cm}

    \textit{\small Autorzy:}\\

    \normalsize \textsc{\@author} \par

    \end{minipage}
         \end{flushright}
 \begin{center}
    \vspace{5cm}

     {\small Praca wykonana w~ramach przedmiotu:
     Języki i~metody programowania 2}

    \vspace*{\stretch{6}}

    \@date

    \end{center}

  \end{titlepage}

}

\makeatother

\title{Specyfikacja implementacyjna}

\title{SPECYFIKACJA IMPLEMENTACYJNA\\projekt: wireworld}

\author{Anna Głowińska\newline numer indeksu: 291070\newline anna.glowinska98@gmail.com \newline \newline
Adam Czajka\newline numer indeksu: 291063 \newline czajka.adam147@gmail.com}

\maketitle
\tableofcontents
\thispagestyle{fancy}
\newpage

\section{Informacje ogólne}

\subsection{Uruchomienie programu}
Uruchomienie programu \verb+Wireworld+ następuje poprzez wciśnięcie przez użytkownika przycisku \verb+run+ w programie \verb+IntelliJ+. Po jego włączeniu program odczytuje dane wejściowe podane uprzednio przez użytkownika i przedstawia upływ czasu w postaci dyskretnych kroków czasowych - kolejnych generacji.
\par Działanie programu \verb+Wireworld+ odbywa się w głównym oknie aplikacji, gdzie przebiega komunikacja między użytkownikiem a programem w celu zatrzymania lub wznowienia działania programu, zmiany częstotliwości z jaką zachodzą kolejne generacje bądź zapisania bieżącej generacji do pliku TXT.


\subsection{Przebieg programu}
Program na podstawie pobranych od użytkownika informacji z pliku TXT wykonuje ze stałą częstotliwością kolejne generacje. Każdorazowo pokazuje obecną generację w głównym oknie aplikacji i~oczekuje na upłyniecie określonego przez użytkownika czasu do wykonania kolejnej generacji. Po wyborze zmiany częstotliwości z paska menu aplikacji następuje zmiana okresu trwania generacji. Po wyborze opcji:
\begin{itemize}
\item \verb+start+ następuje wznowienie działania automatu komórkowego,
\item \verb+stop+ następuje zatrzymanie działania automatu komórkowego,
\item \verb+zapisz+ następuje zapisanie bieżącej generacji do pliku TXT.
\end{itemize}

\newpage
\section{Opis modułów}

\subsection{Diagram modułów}

\begin{figure}[ht]
\centering
\includegraphics[width=1.0\textwidth]{class_java_wireworld.png}
\caption{Diagram modułów.}
\label{fig:k1}
\end{figure}

\subsection{Uzasadnienie}
Diagram modułów programu \verb+Wireworld+ ma postać przedstawioną na rysunku nr 1. Moduły są między sobą powiązane relacją zależności, ponieważ wymagają ich do realizacji własnej funkcjonalności. Zależność między dwoma modułami oznaczona jest strzałką i polega na tym, że moduł w stronę którego skierowany jest grot strzałki korzysta drugiego z danych modułów - zmiana w tym module może spowodować konieczność zmiany w module oznaczonym strzałką.

\subsection{Klasa Cell}

Klasa \verb+Cell+ przechowuje stan w jakim aktualnie się znajduje oraz posiada metody \verb+setCellState+, \verb+getCellState+ i \verb+getStateColor+.

\begin{description}
\item[Metoda \texttt{setCellState}] \hfill \\
Metoda odpowiada za ustawienie stanu komórki.
\begin{itemize}
\item metoda przyjmuje jako argument \verb+CellState+,
\item metoda jest typu \verb+void+ - nie przekazuje informacji zwrotnej.
\end{itemize}

\item[Metoda \texttt{getCellState}] \hfill \\
Metoda zwraca stan komórki.
\begin{itemize}
\item metoda jest bezargumentowa,
\item metoda zwraca \verb+CellState+.
\end{itemize}

\item[Metoda \texttt{getStateColor}] \hfill \\
Metoda zwraca odpowiedni kolor dla danego stanu komórki.
\begin{itemize}
\item metoda jest bezargumentowa,
\item metoda zwraca \verb+Color+.
\end{itemize}

\end{description}

\subsection{Klasa Matrix}

Klasa \verb+Matrix+ przechowuje dwuwymiarową tablicę obiektów klasy \verb+Cell+ oraz posiada metody \verb+getCell+, \verb+read+ i \verb+write+.

\begin{description}
\item[Metoda \texttt{getCell}] \hfill \\
Metoda zwraca komórkę o zadanych współrzędnych.
\begin{itemize}
\item metoda przyjmuje jako argumenty dwie liczby całkowite,
\item metoda zwraca obiekt klasy \verb+Cell+.
\end{itemize}

\item[Metoda \texttt{read}] \hfill \\
Metoda wywołuje wielokrotnie metodę \verb+readNextState+ klasy \verb+Parser+, aby utworzyć całą macierz potrzebną do działania programu \verb+Wireworld+.
\begin{itemize}
\item metoda jako argument przyjmuje ścieżkę do pliku,
\item metoda jest typu \verb+void+ - nie przekazuje informacji zwrotnej.
\end{itemize}

\item[Metoda \texttt{write}] \hfill \\
Metoda wywołuje wielokrotnie metodę \verb+writeNextState+ klasy \verb+MyFileWriter+, aby zapisać całą macierz stanowiącą obraz bieżącej generacji do pliku TXT.
\begin{itemize}
\item metoda jako argument przyjmuje ścieżkę do pliku,
\item metoda jest typu \verb+void+ - nie przekazuje informacji zwrotnej.
\end{itemize}

\end{description}

\subsection{Klasa Game}

Klasa \verb+Game+ odpowiada za graficzny interfejs użytkownika oraz posiada metodę \verb+run+.

\begin{description}
\item[Metoda \texttt{run}] \hfill \\
Metoda odpowiada za przeprowadzenie całej sekwencji generacji tworzących symulację \verb+Wireworld+.
\begin{itemize}
\item metoda jest bezargumentowa,
\item metoda jest typu \verb+void+ - nie przekazuje informacji zwrotnej.
\end{itemize}

\end{description}

\subsection{Klasa Parser}

Klasa \verb+Parser+ odpowiada za pobranie danych z pliku wejściowego TXT oraz posiada metodę \verb+readNextState+.

\begin{description}
\item[Metoda \texttt{readNextState}] \hfill \\
Metoda odpowiada za odczyt stanu komórki z pliku TXT.
\begin{itemize}
\item metoda jest bezargumentowa,
\item metoda zwraca \verb+CellState+.
\end{itemize}

\end{description}

\subsection{Klasa MyFileWriter}

Klasa \verb+MyFileWriter+ odpowiada za zapis bieżącej generacji do pliku TXT oraz posiada metodę \verb+getCell+.

\begin{description}
\item[Metoda \texttt{writeNextState}] \hfill \\
Metoda odpowiada za zapis stanu komórki do pliku TXT.
\begin{itemize}
\item metoda przymuje jako argument ścieżkę do pliku oraz \verb+CellState+,
\item metoda jest typu \verb+void+ - nie przekazuje informacji zwrotnej.
\end{itemize}

\end{description}

\subsection{Enum CellState}

Enum \verb+CellState+ to struktura danych służąca do reprezentacji stanu komórek. Wyróżniamy w niej stany: \verb+EMPTY+, \verb+ELECTRON_HEAD+, \verb+ELECTRON_TAIL+ oraz \verb+CONDUCTOR+.

\section{Przechowywanie danych}
\subsection{Struktura danych}

Dane wejściowe są zapisywane w~obiekcie \verb+matrix+ klasy \verb+Matrix+. Obiekt \verb+matrix+ zawiera tablicę obiektów klasy \verb+Cell+ o~wymiarach zgodnych z~danymi wejściowymi.
\par Dostęp do danych znajdujących się w~obiekcie \verb+matrix+ następuje poprzez wywołanie metody \verb+getCell+.

\section{Sprzęt i oprogramowanie}

\subsection{System operacyjny}
Program \verb+Wireworld+ został napisany, kompilowany i~testowany na systemie operacyjnym \verb+macOS High Sierra+. System ten jest z~rodziny \verb+UNIX+.

\subsection{Język programowania}
Program \verb+Wireworld+ został napisany w~języku \verb+Java+.

\subsubsection{Wersja}
Program \textit{Wireworld} został napisany za pomocą programu \verb+IntelliJ+ \verb+IDEA 2017.3.5+ \verb+(Community Edition)+.

\subsection{Testowanie programu}

\subsubsection{Testy}
Do testowania używane są pliki testowe zawierające testy pojedynczych funkcjonalności programu. Wszystkie pliki zawierające testy przechowywane są w~katalogu \verb+wireworld/src/ww/tests+ i~nazwane zgodnie z~następującą koncepcją: \verb+,,test_co-jest-testowane_co-powinno-być-wynikiem/skutkiem.java''+.

\par
Testy jednostkowe pisane są na zasadzie given-when-then. Oznacza to, że kod każdego testu zostanie podzielony na trzy części:
\begin{itemize}
\item given - odpowiadającą za dane, które test pobiera z~programu,
\item when - wykonująca odpowiednie działania, wywołująca funkcje etc.,
\item then - sprawdzająca poprawność wykonanych działań, wywołanych funkcji etc.
\end{itemize}

\section{Wersjonowanie}

\subsection{Branch'e}
Program jest pisany i~commit'owany z~jednej tylko gałęzi \verb+master+, ponieważ w~dwuosobowej zajmującej się tym programem grupie zachowane zostają zasady extreme programming zakładające obecność dwóch osób podczas pisania kodu.

\subsection{Historia wersji}
Historia wersji programu zostaje wypisana na ekran za pomocą dekstopu \verb+GitCracken+.


\end{document}
